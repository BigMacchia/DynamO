\documentclass[aps,pre,onecolumn,preprint,showpacs]{revtex4}
\usepackage{times,mathptmx,amssymb}
\usepackage{graphicx}
\usepackage{amsmath}    % need for subequations


\begin{document}

\title{Cooling and agreggation in sticky granulates}

\author{S. Gonz\'alez, A.R. Thornton, S. Luding}
\affiliation{Multi Scale Mechanics, CTW, UTwente}
\pacs{
45.70.-n,	%Granular systems
}


\begin{abstract}

  Motivated by nanoaerosols where van der Walls force acts as glue
  between particles, a cluster based event-driven algorithm is
  developed to simulate the formation of clusters in a 3D gas with
  periodic boundary conditions: particles move freely until they
  collide and stick together or break, depending on the energy of the
 interaction.

\end{abstract}
\maketitle

\section{Capsules}

Since in DYNAMO the dynamics of hard lines is already implemented we
decided to start to understand the dynamics of nonsymetric bodies
starting with capusles. A capsules is a cylindrical section with two
half spheres in the extrema. 

The equation to solve for the collision time between hard lines is
anologous the one needed in the cylindrical part of the capsule. For
the two half spheres the problem is a bit more complex. In the
following we use Frenkel-Maguire's notation  \cite{frenkel}. 

One can write as vectors the equation for the collision time 
between two off center spheres. 

\begin{equation}
  \label{eq:f}
  f(t_c) \equiv \|\mathbf{r}_{ij} \pm \mathbf{u}_{i}L/2 \pm \mathbf{u}_{j}L/2\| - d = 0~,
\end{equation}
where $\mathbf{r}_{ij}$ is the distance between the centers of mass of
the capsules, $\mathbf{u}$ is the unitary vector that signals the
position of the two half spheres at $\pm L/2,$ and $d$ is the diameter 
of the spheres. In order to use Frenkel and Maguire's algorithm, we need to 
calculate the first two derivatives of the distance function.

We must remember that 
\begin{equation}
  \label{eq:xprime}
  \dot{\|\mathbf{x}\|}  = \frac{\mathbf{x}\cdotp\dot{\mathbf{x}}}{\|\mathbf{x}\|}~,
\end{equation}
and
\begin{equation}
    \label{eq:x2prime}
  \ddot{\|\mathbf{x}\|}  = 
  \frac{(\dot{\mathbf{x}\cdotp\dot{\mathbf{x}}})\|\mathbf{x}\| - \mathbf{x}\cdotp\dot{\mathbf{x}}\dot{\|\mathbf{x}\|} }{\|\mathbf{x}\|^2}
  = \frac{(\dot{\mathbf{x}}^2 + \mathbf{x}\cdotp\ddot{\mathbf{x}})\|\mathbf{x}\| - \mathbf{x}\cdotp\dot{\mathbf{x}}\dot{\|\mathbf{x}\|} }{\|\mathbf{x}\|^2}~.
\end{equation}

Using the Eq. \eqref{eq:xprime} one can rewrite the second derivative as the sum of three terms,
\begin{equation}
  \label{eq:x2primeb}
  \ddot{\|\mathbf{x}\|}  =  \frac{\dot{\mathbf{x}}^2 }{\|\mathbf{x}\|^2} +  \frac{  \mathbf{x}\cdotp\ddot{\mathbf{x}} }{\|\mathbf{x}\|} - \frac{(\mathbf{x}\cdotp\dot{\mathbf{x}})^2 }{\|\mathbf{x}\|^3} ~.
\end{equation}

The derivatives of the ceters' position are easy to calculate. Indeed,
$\dot{\mathbf{r}}_{ij} = \mathbf{v}_{ij}~,$ while
$\ddot{\mathbf{r}}_{ij} = 0~,$ since we assume that both particles
are under the action of gravity.

By definition one has that $\dot{\mathbf{u}} =
\mathbf{\omega}\times\mathbf{u}.$ Since $\mathbf{u}$ is a unitary vector, the
second derivative of it will be $\ddot{\mathbf{u}} =
\dot{\mathbf{\omega}}\times\mathbf{u} + \mathbf{\omega}\times\dot{\mathbf{u}} = 
\dot{\mathbf{\omega}}\times\mathbf{u} - \omega^2\mathbf{u}.$ In the case of the capsule, the
vector $\mathbf{u}$ coincides with one of the principal axes of the body,
so the angular aceleration can be easily written using Euler's
equations in the solid body frame of reference and in the abscence of 
torques

\begin{eqnarray}
I_1\dot{\tilde{\omega}}_{1}+(I_3-I_2)\tilde{\omega}_2\tilde{\omega}_3 &=& 0~,\nonumber\\
I_2\dot{\tilde{\omega}}_{2}+(I_1-I_3)\tilde{\omega}_3\tilde{\omega}_1 &=& 0~,\\
I_3\dot{\tilde{\omega}}_{3}+(I_2-I_1)\tilde{\omega}_1\tilde{\omega}_2 &=& 0~,\nonumber
\end{eqnarray}
where $I_1$, $I_2$, and $I_3$ are the principal components of the
moment of inertia tensor and $\tilde{\omega}_i$ are the three
components of the angular velocity in the solid body frame.

With this, one can write the first derivative of the indicator function as 
\begin{eqnarray}
\dot{f} &=&
\frac{(\mathbf{r}_{ij} \pm \mathbf{u}_{i}L/2 \pm \mathbf{u}_{j}L/2)\cdotp(\mathbf{v}_{ij} \pm \mathbf{\omega}_i\times\mathbf{u}_{i}L/2 \pm \mathbf{\omega}_j\times\mathbf{u}_{j}L/2)}{\|\mathbf{r}_{ij} \pm \mathbf{u}_{i}L/2 \pm \mathbf{u}_{j}L/2\|}~.
\end{eqnarray}

The second derivative is more cumbersome but nevertheless straightforward to write down when following Eq. \eqref{eq:x2primeb}

\begin{eqnarray}
  \label{eq:ddotf}
 \nonumber  \ddot{f} &=&  \frac{\|\mathbf{v}_{ij} \pm \mathbf{\omega}_i\times\mathbf{u}_{i}L/2 \pm \mathbf{\omega}_j\times\mathbf{u}_{j}L/2\|^2}{\|\mathbf{r}_{ij} \pm \mathbf{u}_{i}L/2 \pm \mathbf{u}_{j}L/2\|} \\
&+&  \frac{ (\mathbf{r}_{ij} \pm \mathbf{u}_{i}L/2 \pm \mathbf{u}_{j}L/2)\cdotp(\pm (\dot{\mathbf{\omega}}_i\times\mathbf{u}_i - \omega_i^2\mathbf{u}_i) \pm (\dot{\mathbf{\omega}}_j\times\mathbf{u}_j - \omega_j^2\mathbf{u}_j))L/2}{\|\mathbf{r}_{ij} \pm \mathbf{u}_{i}L/2 \pm \mathbf{u}_{j}L/2\|} \\
 \nonumber &-& \frac{\{(\mathbf{r}_{ij} \pm \mathbf{u}_{i}L/2 \pm \mathbf{u}_{j}L/2)\cdotp(\mathbf{v}_{ij} \pm \mathbf{\omega}_i\times\mathbf{u}_{i}L/2 \pm \mathbf{\omega}_j\times\mathbf{u}_{j}L/2)\}^2}{\|\mathbf{r}_{ij} \pm \mathbf{u}_{i}L/2 \pm \mathbf{u}_{j}L/2\|^3} ~.
\end{eqnarray}


If we consider the simple case of a symmetric top, the angular
velocity in the laboratory frame of reference is constant. This
slightly simplifies Eq. \eqref{eq:ddotf} (making the implementation of
it not dependent on the angular acceleration, which for the case of
the asymmetric body can only be written using special functions):
\begin{eqnarray}
  \label{eq:ddotfST}
  \nonumber \ddot{f} &=&  \frac{\|\mathbf{v}_{ij} \pm \mathbf{\omega}_i\times\mathbf{u}_{i}L/2 \pm \mathbf{\omega}_j\times\mathbf{u}_{j}L/2\|^2}{\|\mathbf{r}_{ij} \pm \mathbf{u}_{i}L/2 \pm \mathbf{u}_{j}L/2\|} \\
&+&  \frac{ (\mathbf{r}_{ij} \pm \mathbf{u}_{i}L/2 \pm \mathbf{u}_{j}L/2)\cdotp(\mp  \omega_i^2\mathbf{u}_i \mp  \omega_j^2\mathbf{u}_j)L/2}{\|\mathbf{r}_{ij} \pm \mathbf{u}_{i}L/2 \pm \mathbf{u}_{j}L/2\|} \\
&-& \nonumber \frac{\{(\mathbf{r}_{ij} \pm \mathbf{u}_{i}L/2 \pm \mathbf{u}_{j}L/2)\cdotp(\mathbf{v}_{ij} \pm \mathbf{\omega}_i\times\mathbf{u}_{i}L/2 \pm \mathbf{\omega}_j\times\mathbf{u}_{j}L/2)\}^2}{\|\mathbf{r}_{ij} \pm \mathbf{u}_{i}L/2 \pm \mathbf{u}_{j}L/2\|^3} ~.
\end{eqnarray}

In the laboratory frame of reference, the orientation vector will have
a temporal dependence of the form \[\mathbf{u}_i =
\left( {\begin{array}{cc}
x(t) \\
y(t) \\
z(t) \\
 \end{array} } \right) ~,\] where $\omega$ is the
angular velocity and $\delta$ is the initial angle at the last
collision, from where the time $t$ is measured. With this, one can
derive an upper bound for the first and second derivative easily. We
need this in order to ``obtain improved bounds on the interval within
which collisions may occur'' \cite{frenkel}.


Naturally, all the aforementioned problem with the norm of the vector can be avoided using the
square of the  collision time:

\begin{equation}
  \label{eq:f}
  f(t_c) \equiv \|\mathbf{r}_{ij} \pm \mathbf{u}_{i}L/2 \pm \mathbf{u}_{j}L/2\| - d = 0~,
\end{equation}

\begin{thebibliography}{}

\bibitem{frenkel}  D. Frenkel, J. F. Maguire, Mol. Phys. 49, 503 (1983).


\end{thebibliography}


\end{document}
