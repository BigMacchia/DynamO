\documentclass[aps,pre,onecolumn,preprint,showpacs]{revtex4}
\usepackage{times,mathptmx,amssymb}
\usepackage{graphicx}
\usepackage{amsmath}    % need for subequations

\begin{document}

\title{Sleeping of particles in Dynamo}

\author{S. Gonz\'alez}
\affiliation{Multi Scale Mechanics, CTW, UTwente}
\pacs{
45.70.-n,	%Granular systems
}


\begin{abstract}

  The Sleeping of particles in Dynamo is explained.

\end{abstract}
\maketitle

\section{Conditions}


A particle is asleep if its velocity is smaller than a threshold value
$V_{\rm sleep}$ and it is in a mechanically stable position. To check
this last condition, without making a detailed force analysis, which
is anyway ill defined in hard sphere models, we propose a heuristic
method. A particle approaching a mechanically stable position proceeds
with a series of collisions with other particles where, at least one
of them, is already a frozen particle. When approaching this
equilibrium position the velocity must be decreasing but, to avoid to
erroneously identify a particle that is climbing over another, it is
also demanded that the particle is going down. In summary a particle
is asleep if when colliding with a frozen particle: (a) it is going
down and (b) its velocity is smaller than at the previous collision
and smaller than $V_{\rm sleep}$. To avoid sleeping a particle at
exactly contact with another, which would lead to ill defined dynamics
when awake, it is asleep in an advanced position (halfway to the next
collision that it would have had if it continues with the normal
dynamics).  Sleeping particles partially solves one of drawbacks of
the IHS model: frozen particles have multiple contacts but normal ones
only have binary interactions.

Sleeping a particle is an inelastic process because its energy,
although small, is lost. If there is a sequence of repeated sleeping
and waking up processes, an inelastic collapse can occur. This
sequence can take place when a particle that is on top of a frozen
region is collided frequently from above as in avalanches or in a
deposition over a surface.  To avoid this inelastic collapse, the
energy that the particle had when it was asleep is stored for a time
of the order of $2V_{\rm sleep}/(1-\alpha_{\rm frozen})g$, that is,
the inelastic collapse time over a frozen region.  If the particle is
awaken before that time, this energy needs to be reinjected.
\begin{thebibliography}{}




\end{thebibliography}


\end{document}
